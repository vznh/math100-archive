\documentclass{article}
\usepackage{amsmath} % For better math support
\usepackage{amssymb}


\title{MATH 100 - PSet \#6}
\author{JasonVinh Son}
\date{November 2024}

\begin{document}

\maketitle

\section*{6.2.5}
Let \( a, b, c \in \mathbb{Z} \). Prove or disprove: If \( a \mid bc \), then \( a \mid b \) or \( a \mid c \).

\section*{Solution}
This statement is false. Disproving by counterexample:

Consider \( a = 6 \), \( b = 2 \), and \( c = 3 \).

\begin{itemize}
    \item Calculate \( bc \):
    \[
    bc = 2 \times 3 = 6.
    \]
    \( 6 \mid 6 \) holds, so \( a \mid bc \).

    \item Now, we'll check if \( a \mid b \) or \( a \mid c \):
    \begin{align*}
        6 \mid 2 & \quad \text{(False, as } \frac{2}{6} \text{ is not an integer)} \\
        6 \mid 3 & \quad \text{(False, as } \frac{3}{6} \text{ is not an integer)}
    \end{align*}
\end{itemize}

Thus, \( a \mid bc \) is true, but neither \( a \mid b \) nor \( a \mid c \) is true. This counterexample shows that the original statement is \textbf{false}.

\medskip

\section*{6.2.14}
Assume that \( r \) and \( s \) are relatively prime. Show that \( r + s \) and \( s \) are relatively prime also.

\section*{Solution}
Given that \( r \) and \( s \) are relatively prime, we have:
\[
\gcd(r, s) = 1.
\]

We want to show that \( \gcd(r + s, s) = 1 \).

\begin{enumerate}
    \item Let \( d \) be the greatest common divisor of \( r + s \) and \( s \). Thus,
    \[
    d \mid (r + s) \quad \text{and} \quad d \mid s.
    \]

    \item If \( d \mid s \), we can express \( s = d k \) for some integer \( k \).

    \item Since \( d \mid (r + s) \), we have:
    \[
    r + s = d m \quad \text{for some integer } m.
    \]

    \item Subtracting the two expressions:
    \[
    r = d (m - k),
    \]
    implies that \( d \mid r \).

    \item Thus, \( d \) divides both \( r \) and \( s \). However, since \( \gcd(r, s) = 1 \) by assumption, the only integer that divides both is \( d = 1 \).
\end{enumerate}

Therefore,
\[
\gcd(r + s, s) = 1,
\]
showing that \( r + s \) and \( s \) are relatively prime.

\section*{6.2.18}
Let \( b, c \in \mathbb{Z} \). Suppose that \( b \) and \( c \) are relatively prime. Show that for all integers \( a \), \(\gcd(a, b)\) and \(\gcd(a, c)\) are relatively prime.

\section*{Solution}
Let \( d_1 = \gcd(a, b) \) and \( d_2 = \gcd(a, c) \). We want to show that:
\[
\gcd(d_1, d_2) = 1.
\]

\begin{enumerate}
    \item Let \( d = \gcd(d_1, d_2) \). By definition, \( d \mid d_1 \) and \( d \mid d_2 \), which implies:
    \[
    d \mid \gcd(a, b) \quad \text{and} \quad d \mid \gcd(a, c).
    \]

    \item Since \( d \mid d_1 \), we have \( d \mid a \) and \( d \mid b \). Similarly, since \( d \mid d_2 \), we have \( d \mid a \) and \( d \mid c \).

    \item Therefore, \( d \) divides both \( b \) and \( c \).

    \item Given that \( \gcd(b, c) = 1 \), the only integer that divides both \( b \) and \( c \) is \( d = 1 \).
\end{enumerate}

Thus, we have shown that:
\[
\gcd(\gcd(a, b), \gcd(a, c)) = 1.
\]

Therefore, \( \gcd(a, b) \) and \( \gcd(a, c) \) are relatively prime for all integers \( a \).


\medskip

\section*{6.3.3 (1, 2)}
Use the Euclidean algorithm to find the greatest common divisor for the following pairs of numbers:

\begin{enumerate}
    \item \( 48 \) and \( 18 \)
    \item \( 2,700 \) and \( 17,640 \)
\end{enumerate}

\section*{Solution}

\textbf{For \(\gcd(48, 18)\):}

\[
48 = 18(2) + 12
\]
\[
18 = 12(1) + 6
\]
\[
12 = 6(2) + 0
\]

Thus, \(\gcd(48, 18) = 6\).

\medskip

\textbf{For \(\gcd(2,700, 17,640)\):}

\[
17,640 = 2,700(6) + 1,440
\]
\[
2,700 = 1,440(1) + 1,260
\]
\[
1,440 = 1,260(1) + 180
\]
\[
1,260 = 180(7) + 0
\]

Thus, \(\gcd(2,700, 17,640) = 180\).

\medskip


\section*{6.4.3}
Let \( a \) and \( b \) be relatively prime integers. Let \( c \) be any integer. Prove the following:

\begin{enumerate}
    \item If \( a \mid bc \), then \( a \mid c \).
    \item If \( a \mid c \) and \( b \mid c \), then \( ab \mid c \).
\end{enumerate}

\section*{Solution}

\begin{enumerate}
    \item \textbf{If \( a \mid bc \), then \( a \mid c \)}

    \begin{enumerate}
        \item Given: \( a \mid bc \) and \( \gcd(a, b) = 1 \).
        \item Since \( a \mid bc \), there exists an integer \( k \) such that:
        \[
        bc = ak.
        \]
        \item Since \( \gcd(a, b) = 1 \), \( a \) and \( b \) have no common factors other than 1.
        \item By a property of relatively prime integers, if \( a \mid (bc) \) and \( \gcd(a, b) = 1 \), then \( a \mid c \).
    \end{enumerate}
    Thus, \( a \mid c \).

    \item \textbf{If \( a \mid c \) and \( b \mid c \), then \( ab \mid c \)}

    \begin{enumerate}
        \item Given: \( a \mid c \) and \( b \mid c \) with \( \gcd(a, b) = 1 \).
        \item Since \( a \mid c \), there exists an integer \( m \) such that:
        \[
        c = am.
        \]
        \item Since \( b \mid c \), there exists an integer \( n \) such that:
        \[
        c = bn.
        \]
        \item Thus, we have:
        \[
        am = bn.
        \]
        \item Dividing both sides by \( a \), we get:
        \[
        m = \frac{bn}{a}.
        \]
        \item Since \( \gcd(a, b) = 1 \), \( a \) divides \( n \). Let \( n = ak \) for some integer \( k \).
        \item Substituting \( n = ak \) into the equation \( c = bn \), we get:
        \[
        c = b(ak) = abk.
        \]
    \end{enumerate}
    Thus, \( ab \mid c \).
\end{enumerate}
\end{document}
